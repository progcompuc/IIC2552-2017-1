\documentclass[10pt,letterpaper]{extarticle}
\usepackage{logo}
\usepackage{fullpage}
\usepackage[utf8]{inputenc}
\usepackage[spanish]{babel}
\usepackage[hidelinks]{hyperref}
\usepackage{epsfig}
\usepackage{amsmath}
\usepackage{amssymb}
\usepackage{gensymb}
\usepackage{multicol}
\usepackage{color}
\usepackage{hhline}
\usepackage{epstopdf}
\usepackage{tikz}
\usepackage[sol]{optional}

\newcommand{\N}{\mathbb{N}}
\newcommand{\Exp}[1]{\mathcal{E}_{#1}}
\newcommand{\List}[1]{\mathcal{L}_{#1}}
\newcommand{\Array}[1]{\mathcal{A}_{#1}}
\newcommand{\EN}{\Exp{\N}}
\newcommand{\LN}{\List{\N}}
\newcommand{\AB}{\Array{B}}
\newcommand{\AN}{\Array{\N}}

\newcommand{\comment}[1]{}
\newcommand{\lb}{\\~\\}
\newcommand{\eop}{_{\square}}
\newcommand{\hsig}{\hat{\sigma}}
\newcommand{\ra}{\rightarrow}
\newcommand{\lra}{\leftrightarrow}
\newcommand{\ar}[2]{\big[#1 : #2\big]}

\newcommand{\twopartdef}[4]
{
	\left\{
		\begin{array}{ll}
			#1 &  #2 \\
			#3 &  #4
		\end{array}
	\right.
}

\begin{document}
\begin{tabular}{ccl}
\begin{tabular}{c}
\psfig{file=puclogo.eps}
\end{tabular}
&\ \ \ &
\begin{tabular}{l}
PONTIFICIA UNIVERSIDAD CATOLICA DE CHILE\\
ESCUELA DE INGENIERIA\\
DEPARTAMENTO DE CIENCIA DE LA COMPUTACION
\end{tabular}
\end{tabular}

\begin{center}
\bf IIC2552 - Taller de Programación Avanzada\\
\bf 1\degree semestre 2017

\vspace{0.5cm}

\bf {\Huge Programa de curso}
\end{center}

\vspace{4ex}

\begin{center}
\begin{tabular}{l l l c l}
	\textbf{Profesor} : & Gabriel
	Diéguez (\href{mailto:gsdieguez@ing.puc.cl}{gsdieguez@ing.puc.cl}) &
	\textbf{Horario cátedra} & : & V:4-6, sala Javier Pinto\\
	\textbf{Ayudante} : & Pablo Messina
	(\href{mailto:pamessina@uc.cl}{pamessina@uc.cl}) &
	\textbf{Sitio Web} & : & Siding y Github
\end{tabular}
\end{center}

\section*{Objetivos}

\noindent
El objetivo de este curso es que los alumnos desarrollen y apliquen los conocimientos
aprendidos en los cursos de programación en el contexto de Programación Competitiva.
Al finalizar el curso, el alumno habrá adquirido la destreza necesaria para
resolver en un tiempo corto problemas de programación, aplicando algunas
técnicas estándares. Adicionalmente, se espera preparar equipos para participar
en el concurso de programación que organiza anualmente la ACM.\\

\section*{Contenidos}
\begin{enumerate}
  \item Técnicas de diseño de algoritmos: backtracking, programación dinámica,
	algoritmos codiciosos, dividir y conquistar.
	\item Estructuras de datos: listas, árboles, grafos, arreglos.
	\item Técnicas de programación: input/output, librerías estándar.
	\item Algoritmos de uso frecuente: Dijkstra, Binary/Ternary search, BFS, DFS, otros.
	\item Otros problemas: strings, matemáticas, geometría.
\end{enumerate}

\section*{Metodología}
\noindent
El curso se basa en clases teórico-prácticas, en que los alumnos deben desarrollar
un set de problemas. En algunas ocasiones se tratará un tópico específico, habiendo una
pequeña clase al principio, para luego resolver un set de problemas relacionado.\\

\noindent
Los problemas son evaluados por jueces automáticos, y pueden ser resueltos en grupos
de no más de 3 personas, debiendo cada alumno enviar sus respuestas por separado.\\

\noindent
Los lenguajes de programación aceptados son C++, Java y Python.

\section*{Evaluación}
\noindent
La evaluación del curso se basa en la asistencia y participación en clases. Se
considerará presente a quien:

\begin{enumerate}
	\item Resuelva al menos un problema durante las clases teórico-prácticas, o
	\item Resuelva al menos dos problemas durante las clases 100\% prácticas, o
	\item Resuelva al menos tres problemas de manera remota (los contests estarán disponibles durante toda la semana).
\end{enumerate}

Tendrá un 7 quien tenga un 100\% de asistencia. Si no, se calculará su nota proporcionalmente
a la cantidad de asistencias según los criterios anteriores.

\section*{Bibliografía}
\noindent
Todo el material se encuentra disponible en las páginas del curso.

\section*{POLÍTICA DE INTEGRIDAD ACADÉMICA DEL DEPARTAMENTO DE CIENCIA DE LA
COMPUTACIÓN} Los alumnos de la Escuela de Ingeniería de la Pontificia
Universidad Católica de Chile deben mantener un comportamiento acorde a la
Declaración de Principios de la Universidad.  En particular, se espera que
mantengan altos estándares de honestidad académica.  Cualquier acto deshonesto
o fraude académico está prohibido; los alumnos que incurran en este tipo de
acciones se exponen a un Procedimiento Sumario. Es responsabilidad de cada
alumno conocer y respetar el documento sobre Integridad Académica publicado por
la Dirección de Docencia de la Escuela de Ingeniería.\\

Específicamente, para los cursos del Departamento de Ciencia de la Computación,
rige obligatoriamente la siguiente \emph{política de integridad académica}.
Todo trabajo presentado por un alumno para los efectos de la evaluación de un
curso debe ser hecho \textbf{individualmente} por el alumno, \textbf{sin apoyo
en material de terceros}.  Por ``trabajo'' se entiende en general las
interrogaciones escritas, las tareas de programación u otras, los trabajos de
laboratorio, los proyectos, el examen, entre otros.\\

En particular, si un alumno copia un trabajo, o si a un alumno se le prueba que
compró o intentó comprar un trabajo, \textbf{obtendrá nota final 1,1 en el
curso} y se solicitará a la Dirección de Docencia de la Escuela de Ingeniería
que no le permita retirar el curso de la carga académica semestral.\\

Por ``copia'' se entiende incluir en el trabajo presentado como propio, partes
hechas por otra persona.  En caso que corresponda a ``copia'' a otros alumnos,
la sanción anterior se aplicará a todos los involucrados.  En todos los casos,
se informará a la Dirección de Docencia de la Escuela de Ingeniería para que
tome sanciones adicionales si lo estima conveniente.\\

Obviamente, está permitido usar material disponible públicamente, por ejemplo,
libros o contenidos tomados de Internet, \textbf{siempre y cuando se incluya la
referencia correspondiente}.\\

Lo anterior se entiende como complemento al Reglamento del Alumno de la
Pontificia Universidad Católica de Chile
(\href{http://www.ing.puc.cl/alumnos/reglamento/}{www.ing.puc.cl/alumnos/reglamento/}).
Por ello, es posible pedir a la Universidad la aplicación de sanciones
adicionales especificadas en dicho reglamento.

\end{document}
